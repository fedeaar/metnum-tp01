% === demo 1 === %
\subsection{A: $A = pWD + ez^t$}\label{A.1}
\begin{proof}[demostración] 
    
Recordemos que:

\begin{align*}
    e_i     &=  1
    \\
    \\
    z_{j}   &=  \left\{ 
                    \begin{array}{lcc}
                    (1 - p) / n     &  \ \text{si}    &  c_j \neq 0 \\
                    1 / n           &  \ \text{si no} &
                    \end{array}
                \right.\
    \\
    \\
    w_{ij}  &=  \left\{ 
                    \begin{array}{lcc}
                    1               &  \qquad \qquad \text{si}    & i \neq j\  \wedge\ j \stackrel{l}{\longrightarrow} i \\
                    0               &  \qquad \qquad \text{si no} &
                    \end{array}
                \right.\
    \\
    \\
    d_{ij}  &=  \left\{ 
                    \begin{array}{lcc}
                    1 / c_j         &  \qquad \: \: \text{si}    & i = j\  \wedge\ c_j \neq 0 \\
                    0               &  \qquad \ \  \text{si no} &
                    \end{array}
                \right.\
\end{align*}
\vspace{1em}

\noindent A partir de estas definiciones, vemos que, como $\textbf{D}$ es diagonal, el producto a derecha $\textbf{W}\textbf{D}$ escala cada columna $w_j$ por el factor $d_{jj}$, tal que:
\vspace{1em}

\begin {equation*}
    (\textbf{W}\textbf{D})_{ij}  =  \left\{ 
                    \begin{array}{lcc}
                    w_{ij} / c_j    & \ \ \ \ \ \text{si}    & c_j \neq 0 \\
                    0               & \ \ \ \ \ \text{si no} &
                    \end{array}
                \right.\
\end {equation*}
\vspace{1em}

\noindent Como $p$ es un escalar, sigue entonces que:
\vspace{1em}

\begin {equation*}
    (p\textbf{W}\textbf{D})_{ij}  =   \left\{ 
                        \begin{array}{lcc}
                        p \cdot w_{ij} / c_j    &  \text{si}    & c_j \neq 0 \\
                        0                       &  \text{si no} &
                        \end{array}
                    \right.\
\end {equation*}
\vspace{1em}

\noindent Además, $\ e \in \mathbb{R}^{n \times 1}\ \wedge\ z^t \in \mathbb{R}^{1 \times n} \implies ez^t \in \mathbb{R}^{n \times n}\ $, y:
\vspace{1em}

\begin {equation*}
    (ez^t)_{ij} := \sum_{k=1}^{1} e_{ik} \cdot z^t_{kj} = e_i \cdot z^t_j = 1 \cdot z^t_j = z_j 
\end {equation*}
\vspace{1em}

\noindent Por lo que:
\begin {align*}
    (p\textbf{W}\textbf{D} + ez^t)_{ij}   &=   \left\{ 
                                \begin{array}{lcc}
                                p \cdot w_{ij} / c_j + z_j   &  \ \ \ \ \qquad \text{si}    & c_j \neq 0 \\
                                z_j                          &  \ \ \ \ \qquad \text{si no} &
                                \end{array}
                            \right.\ \\
                            \\
                        &=  \left\{ 
                                \begin{array}{lcc}
                                (1 - p) \cdot \frac{1}{n} + p \cdot \frac{w_{ij}}{c_j}   &  \ \ \   \text{si}    & c_j \neq 0 \\
                                \frac {1}{n}                                             &  \ \ \   \text{si no} &
                                \end{array}
                            \right.\
\end {align*}

\noindent pero: 
\vspace{1em}

\begin{equation*}
    a_{ij} := Pr(j \longrightarrow i) = \left\{ 
                                            \begin{array}{lcc}
                                            (1 - p)\cdot \frac{1}{n} + p \cdot \frac{I_{ij}}{c_j}      &  \text{si}    & c_j \neq 0\\
                                            \frac{1}{n}                                                &  \text{si no}  &
                                            \end{array}
                                        \right.
\end{equation*}
\vspace{1em}

Como $\ I_{ij} = 1\ $ si y sólo si existe un hipervínculo de $j$ a $i$, con $j \neq i$ ---y nulo en caso contrario---, entonces $\ I_{ij} = w_{ij}\ $ y concluímos que $\ a_{ij} = (p\textbf{W}\textbf{D} + ez^t)_{ij}$ $\ \forall i, j:\ 1\ ...\ n\ $,  lo que implica que:
\vspace{1em}

\begin{equation*}
    \textbf{A} = p\textbf{W}\textbf{D} + ez^t
\end{equation*}
\vspace{1em}

\end{proof}



% === demo 2 === %
\newpage
\subsection{B: $I - pWD$ permite la eliminación gaussiana}\label{A.2}

\begin{proof}[Demostración] Demos primero una definición formal del enunciado:  
    
\vspace{2em}
\noindent \textsc{Eliminación Gaussiana}. $\forall \mathbf{A} \in \mathbb{R}^{n \times n},\ \exists!\ m \in \mathbb{N}:\ 1 \leq m \leq n$ tal que: 

\vspace{1em}
\noindent $\forall k \in \mathbb{N}:\ 1 \leq k < m$

\vspace{0.5em}
\begin{align*}
    eg_{1}(\mathbf{A})_{ij}  &= a_{ij}
    \\
    eg_{k+1}(\mathbf{A})_{ij} &=  
        \left\{ 
            \begin{array}{lcc}
                eg_{k}(\mathbf{A})_{ij}  &   \text{si}    & i < k+1 \ \vee \  j < k+1 \\
                eg_{k}(\mathbf{A})_{ij} - \frac {eg_{k}(\mathbf{A})_{ik}}{eg_{k}(\mathbf{A})_{kk}}\cdot \ eg_{k}(\mathbf{A})_{kj} & \text{si no} \\
            \end{array}
        \right.\ 
\end{align*}  

\vspace{1em}    
\noindent donde $m < n$ es el mínimo valor que satisface que $eg_{m}{(\mathbf{A})}_{mm} = 0$. Si la matríz $\mathbf{A}$ es tal que $m = n$, entonces $\mathbf{A}$ permite la eliminación gaussiana.

\vspace{3em}
\noindent Probaremos que para la matríz $\ \mathbf{B} \in \mathbb{R}^{n \times n} = \mathbf{I} - p\mathbf{W}\mathbf{D}\ $, $\ \forall k \in \mathbb{N}: 1 \leq k \leq n\ $: 

\begin{equation*}
eg_{k}{(\mathbf{B})}_{kk} \neq 0
\end{equation*}

\vspace{0.5em}
\noindent y en consecuencia $m = n$.


\vspace{2em}
\noindent Para ello demostraremos el siguiente enunciado:

\begin{equation*}
    |b^{k}_{jj}| \ > \sum_{i = k,\ i \neq j}^{n} |b^{k}_{ij}| \qquad \forall k,\ j \in \mathbb{N}: 1 \leq k \leq j \leq n
\end{equation*}

\vspace{0.5em}
\noindent donde $b^k_{ij} = eg_k(\mathbf{B})_{ij}$.

\vspace{1em}
Es decir, para todo paso $k$ de la eliminación gaussiana se satisface que el valor absoluto de los elementos de la diagonal mayores o iguales a $k$ son mayor estricto al valor absoluto de la suma de los elementos de su columna desde la posición $k$, descontando la diagonal.

\vspace{1em}
\noindent Lo demostraremos por inducción:

\vspace{2em}
\noindent $\forall k,\ j \in \mathbb{N}: 1 \leq k \leq j \leq n$ 

\begin{equation*}
    |b^{k}_{jj}| \ > \sum_{i = k,\ i \neq j}^{n} |b^{k}_{ij}| \implies |b^{k+1}_{jj}| \ > \sum_{i = k+1,\ i \neq j}^{n} |b^{k+1}_{ij}|
\end{equation*}

\newpage
\noindent \textsc{Caso base} ($k = 1$):

\vspace{1em}
\noindent Notemos primero que 

\begin{equation*}
    (\mathbf{B})_{ij} = (\mathbf{I} - p\mathbf{W}\mathbf{D})_{ij} =
    \left\{ 
        \begin{array}{lcc}
            1           & \text{si} & i = j   \\
            -p / c_j    & \text{si} & w_{ij} = 1\ \wedge\ c_j \neq 0 \\
            0           & \text{si no}   
        \end{array}
    \right.\ 
\end{equation*}

\vspace{2em}
Dado que todo elemento de cada columna $j$ de $\mathbf{W}\mathbf{D}$ está normalizado por $1/c_j$, tal que su suma da uno, entonces la suma de la columna $(-p\mathbf{W}\mathbf{D})_{j} = -p$, con $0 < p < 1$.

\vspace{1em}
Observamos también que la diagonal es nula para esta matríz, tal que su suma por $\mathbf{I}$ da $(\mathbf{I} - p\mathbf{W}\mathbf{D})_{jj} = 1$. Este tipo de matríz se conoce como estocástica en columnas y su transpuesta es diagonal dominante. 

\vspace{1em}
\noindent En consecuencia, es inmediato que $\ |b^{1}_{jj}| = |b_{jj}| = 1\ >\ \sum_{i = k,\ i \neq j}^{n} |b^{1}_{ij}| = |-p|\ \forall j \in \mathbb{N}: 1 \leq j \leq n$. 

\hfill$\square$


\vspace{2em}
\noindent \textsc{Paso inductivo}:

\vspace{1em}
\noindent $\forall j \in \mathbb{N}: 1 \leq k \leq j \leq n$

\begin{align*}
    |b^{k+1}_{jj}|\ 
        &>\ \sum_{i = k+1,\ i \neq j}^{n} |b^{k+1}_{ij}| \\
    \iff |b^k_{jj} - \frac {b^k_{jk} \ \cdot \ b^k_{kj}}{b^k_{kk}}|\     
        &>\ \sum_{i=k+1,\ i \neq j}^{n} |b^k_{ij} - \frac {b^k_{ik} \ \cdot \ b^k_{kj}}{b^k_{kk}}|
\end{align*}

\vspace{1em}
\noindent dado que $|a - b| \geq |a| - |b|\ \wedge\ |a| + |b| \geq |a + b|$ por desigualdad triangular, entonces podemos demostrar equivalentemente, por teorema del sandwich, que:

%\vspace{1em}
\begin{equation*}
    |b^k_{jj}| - |\frac {b^k_{jk} \ \cdot \ b^k_{kj}}{b^k_{kk}}|\  
        >\ \sum_{i=k+1,\ i \neq j}^{n} |b^k_{ij}|\ \ \  +  \sum_{i=k+1,\ i \neq j}^{n} |\frac{b^k_{ik} \ \cdot \ b^k_{kj}}{b^k_{kk}}| \\
\end{equation*} 

\vspace{1em}
\noindent Como:

%\vspace{1em}
\begin{equation*}
    \sum_{i=k+1,\ i \neq j}^{n} |b^k_{ij}|\ 
        =\ (\sum_{i=k, \ i \neq j}^{n} |b^k_{ij}|) -  |b^k_{kj}|
\end{equation*}

\vspace{1em}
\noindent y por hipótesis inductiva:

%\vspace{1em}
\begin{equation*}
    |b^k_{jj}|\
        >\ \sum_{i=k, \ i \neq j}^{n} |b^k_{ij}|\
    \implies\
    |b^k_{jj}| - |b^k_{kj}|\
        >\ (\sum_{i=k, \ i \neq j}^{n} |b^k_{ij}|) -  |b^k_{kj}|
\end{equation*}

\vspace{1em}
\noindent entonces podemos volver a acotar tal que:

%\vspace{1em}
\begin{equation*}
    |b^k_{jj}| - |b^k_{kj}|\ +  \sum_{i=k+1, \ i \neq j}^{n} |\frac{b^k_{ik} \ \cdot \ b^k_{kj}}{b^k_{kk}}|\
        >_{hi}\ \sum_{i=k+1,\ i \neq j}^{n} |b^k_{ij}|\ +  \sum_{i=k+1,\ i \neq j}^{n} |\frac{b^k_{ik} \ \cdot \ b^k_{kj}}{b^k_{kk}}| \\
\end{equation*}

\vspace{1em}
\noindent y considerar que:

%\vspace{1em}
\begin{align*}
    |b^k_{jj}| - |\frac {b^k_{jk} \ \cdot \ b^k_{kj}}{b^k_{kk}}|\ 
        &\geq\ |b^k_{jj}| - |b^k_{kj}| +  \sum_{i=k+1, \ i \neq j}^{n} |\frac{b^k_{ik} \ \cdot \ b^k_{kj}}{b^k_{kk}}|\ \\  
    \iff |b^k_{kj}| - |\frac {b^k_{jk}\ \cdot \ b^k_{kj}}{b^k_{kk}}|\ 
        &\geq\ \frac{ |b^k_{kj}|}{|b^k_{kk}|} \sum_{i=k+1, \ i \neq j}^{N} |b^k_{ik}|
\end{align*}

\vspace{1em}
\noindent Podemos acotar una última vez, tal que:

%\vspace{2em}
\begin{equation*}
    |b^k_{kj}| - |\frac {b^k_{jk}\ \cdot \ b^k_{kj}}{b^k_{kk}}|\ 
        \geq\ \frac{|b^k_{kj}|}{|b^k_{kk}|} (|b^k_{kk}| - |b^k_{jk}|)\
        >_{hi}\ \frac{ |b^k_{kj}|}{|b^k_{kk}|} \sum_{i=k+1, \ i \neq j}^{N} |b^k_{ik}|
\end{equation*}

\vspace{1em}
\noindent Pero:

%\vspace{1em}
\begin{align*}
    |b^k_{kj}| - |\frac {b^k_{jk}\ \cdot \ b^k_{kj}}{b^k_{kk}}|\ 
        \geq\ \frac{|b^k_{kj}|}{|b^k_{kk}|} (|b^k_{kk}| - |b^k_{jk}|)\
    \iff |b^k_{kj}| - |\frac {b^k_{jk}\ \cdot\ b^k_{kj}}{b^k_{kk}}| 
        &\geq  |b^k_{kj}| - |\frac{b^k_{jk}\ \cdot \ b^k_{kj}}{b^k_{kk}}|\\
\end{align*}

\vspace{1em}
\noindent Lo que es trivialmente cierto. 

\hfill$\square$


\vspace{2em}
\noindent Demostramos ahora por absurdo que $b^{k}_{kk} \neq 0\ \ \forall k \in \mathbb{N}: 1 \leq k \leq n\ $.

\vspace{1em}
\noindent Supongamos que existe algún $k$ tal que $b^{k}_{kk} = 0$. Entonces:

%\vspace{1em}
\begin{equation*}
    |b^{k}_{kk}| \ =\ 0\ >\ \sum_{i = k,\ i \neq k}^{n} |b^{k}_{ij}| \qquad \forall j \in \mathbb{N}: 1 \leq k \leq j \leq n
\end{equation*}

\vspace{1em}
\noindent Pero $\sum_{i = k,\ i \neq k}^{n} |b^{k}_{ij}|$ es necesariamente una suma de valores positivos ó $0$. Es decir: 

%\vspace{1em}
\begin{equation*}
    0\ > \sum_{i = k,\ i \neq k}^{n} |b^{k}_{ij}|\ \geq\ 0\ \implies\ 0 > 0
\end{equation*}

\vspace{1em}
\noindent Absurdo! 

\end{proof}
 



\newpage
% === estructuras alternativas === %
\subsection{C: Alternativas de diseño}\label{A.3}

A la hora de decidir cómo representar las matrices se evaluaron las siguientes alternativas:

\vspace{1em}
\begin{itemize}
    \item Como estructura `base', utilizar un vector de $n$ vectores de tamaño fijo $n$. \textsc{Ventajas}: se puede acceder y modificar los elementos en $\Theta(1)$. \textsc{Desventajas}: requiere iterar sobre todos los elementos de la matríz, es ineficiente en espacio y ---para matrices muy grandes--- probablemente requiera más accesos a disco.\\
    \item Una estructura `alt' con un vector de $n$ vectores de tamaño variable donde en cada lugar se guarda la tupla $<posicion, elemento>$. \textsc{Ventajas}: es más eficiente en espacio, permite iterar sobre los elementos no nulos de cada fila y acceder a estas en $\Theta(1)$. \textsc{Desventajas} El acceso a cada columna ---con búsqueda binaria--- tiene un peor caso en $\Theta(log(n))$ y la inserción tiene un peor caso en $\Theta(n)$.\\
    \item La misma estructura pero con listas enlazadas. \textsc{Ventajas}: permite la inserción en $\Theta(1)$ e iterar sobre los elementos no nulos de cada fila con la misma complejidad. \textsc{Desventajas} el acceso es en $\Theta(n)$ y no mantiene memoria contigua, por lo que hay mayor probabilidad de cache-miss.\\
    \item Una `matriz enlazada' donde cada nodo apunta al próximo elemento no nulo de la misma fila y columna. \textsc{Ventajas}: permite iterar sobre los elementos no nulos de cada fila y columna en $\Theta(1)$. \textsc{Desventajas}: las mismas que la lista enlazada.\\
    \item Utilizar alguna estructura de diccionario como $<map>$. Las ventajas y desventajas son similares a la de la estructura alternativa si se mantienen las claves en orden y se pueda iterar sobre ellas en $\Theta(1)$.   
\end{itemize}

\vspace{1em}
Todas estas posibilidades motivaron la elaboración de un código genérico que permitiera el intercambio de las representaciones. La clase $matriz<R>$ definida en $./implementacion/$ provee una interfáz de matríz completa junto a la implementación de muchos de sus métodos. 

Una representación $R$ deberá definir sólo aquellas operaciones que conciernen el acceso a datos. Para lograr esta modularización se tomaron dos decisiones importantes:

\vspace{1em}
\begin{itemize}
    \item Se optó por la transparencia referencial de todas las operaciones. Para las representaciones contempladas, generar una nueva instancia es más rápido que modificar la estructura actual.
    \item Todos los algoritmos están implementados para evitar las operaciones redundantes. Par ello, se optó por que cada representación implemente iteradores específicos.
\end{itemize}

\vspace{1em}
Estas decisiones tienen una clara desventaja: el código provisto por $matriz<r>$ podría ser optimizado para cada representación subyacente. Sin embargo, se priorizó la facilidad de la experimentación sobre la optimalidad del código.

\vspace{1em}
Siguiendo esta metodología, se implementaron dos de las alternativas propuestas: `base' y `alt'.
