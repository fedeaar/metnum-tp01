\vspace{1em}

A lo largo de este informe se articularon una serie de aspectos formales que muestran la solidéz matemática del modelo y el algoritmo de $PageRank$. Se realizó una evaluación en profundidad de la implementación propuesta, y se desarrolló una experimentación respecto a las características cualitativas del mismo. 

Si bien no se pudieron explorar todas las aristas que se hubieran deseado, pudimos formar una imágen amplia de sus cualidades y funcionamiento. 
\vspace{1em}

A simple vista, PageRank aparenta ser un modelo formal y muy bien desarrollado para categorizar las páginas con un sustento matemático que lo respalda, pero entrando en detalle se puede observar que es un modelo imperfecto. Puede parecer que la metodología que utiliza $PageRank$ imparte un ranking justo, pero hemos visto que, cualitativamente, distintos factores predominan a la hora de generar su jerarquía y, en particular, ciertas páginas dominarán las relaciones del resto y podrán ejercer su influencia.  

\vspace{1em}

Si bien hay varios aspectos positivos y deseados como lo demuestran los escenarios de $sin\_links$ ---donde un set de páginas aisladas reciben de manera uniforme su puntaje--- o de $Simetria$ ---donde dos estructuras que están formadas por links entrantes y salientes de igual importancia reciben la misma valoración---, también hay aspectos negativos que permiten manipular el sistema.

\vspace{1em}

A partir de los casos analizados, se vislumbran ciertas estrategías que permitirían influenciar el ranking: generar páginas descartables con links a la que querrámos inflar artificialmente, no poner vínculos a otras páginas para evitar elevar sus rankings, lograr que páginas importantes nos referencien... es decir: el crecimiento de una página podría ser inorgánico, generando un efecto contraproducente, donde muchas páginas que no ofrecen un contenido valorable posean un ranking elevado, simplemente por conocer el mecanismo de crecimiento.

\vspace{1em}
Un último factor pendiente a mencionar es el del valor $p$. Como vimos, el mismo tiene una influencia notable en las cualidades del ranking. Tanto desde el aspecto del error numérico, como de su influencia en la elaboración de los puntajes. Pudimos notar que sirve como un factor atenuante de las ponderaciones y por ello tiene un rol crucial a la hora de lograr una jerarquía lo más justa posible.  

\vspace{1em}
Consideramos que $PageRank$ fue una idea revolucionaria, marcó la industria e influenció la forma en que navegamos la web. Pero consideramos que el sistema quedó obsoleto para la internet actual, donde la cantidad de actores que obran para influenciar los resultados y la disparidad de cualquier ranking genérico con los intereses particulares de cada navegante, lo vuelven ---en nuestra opinión--- poco efectivo para jerarquizar los resultados en comparación a métodos más modernos. 