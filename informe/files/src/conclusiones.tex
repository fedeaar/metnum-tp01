\vspace{1em}

A simple vista, PageRank aparenta ser un modelo formal e ideal para categorizar a las páginas con un sustento matemático que la respalda, pero entrando en detalle se puede observar que es un modelo imperfecto.  

\vspace{1em}

Si bien tiene varios aspectos positivos y deseados como lo demuestra "Sin links" donde un set de páginas aisladas reciben de manera uniforme el peso que determina la probabilidad de que el navegante aleatorio termine en su página, o la "Simetría" donde dos estructuras que estan formadas por links entrantes y salientes de igual importancia reciben la misma valoración en el ranking, o como es el caso de la "Transitividad en Cadena" donde se deja en claro como afecta el peso de un nodo importante apuntando a tu página de manera directa o indirecta. 

\vspace{1em}

El sistema es imperfecto ya que si se diera a conocer al mundo el modo en el que se pondera el peso de cada página, podría ser manipulado, ya sea generando páginas descartables con links a la propia o tener en claro que no poner vínculos a otra página es una buena estrategia para mantener el ranking lo más alto posible. Por ende el crecimiento de una página podría ser totalmente artificial, generando un efecto contraproducente, donde muchas páginas que no ofrecen un contenido valorable posean un ranking elevado, simplemente por conocer el mecanismo de crecimiento, además de causar una navegación en la red para el usuario inorgánica.

\vspace{1em}

En conclusión PageRank es un modelo que tiene muchas cualidades positivas para determinar que página es mas relevante, pero también es sencillamente manipulable por lo que para una empresa que lidera el mercado de navegadores web claramente no sería conveniente para ellos implementarlo teniendo en cuenta que las personas con intenciones maliciosas podrían sacar provecho. Aunque es una idea que en la práctica, donde todavía el mundo de la web no estaba muy desarrollado tiene sentido que haya prosperado.

\vspace{1em}

Fue una idea revolucionaria el hecho de ordenar los resultados y evitarle a la gente filtrar entre mucho contenido poco confiable. Marco la industria y destaco a Google, pero con los avances de la tecnología el sistema quedó obsoleto para la navegación web aunque seguramente se pueda aplicar el modelo a otros ambientes con grafos direccionados.