\vspace{1em}

\subsection{Análisis cuantitativo}\label{cuantitativo} Procederemos a evaluar nuestra implementación de $PageRank$ en C++ acorde a los algoritmos propuestos.

\vspace{2em}
\subsubsection{Error relativo} Medimos el error $|\mathbf{A}x - x|_1$ en función del valor de $p$ para cien grafos generados aleatoriamente. En total, obtuvimos 10,000 mediciones\footnote{El script asociado se puede encontrar en $./experimentos/error\_relativo.py$}.  
\vspace{1em}

\noindent \textsc{Metodología}. Se calculó  $x = PageRank(g,\ p)$ y se midió el error relativo $|\mathbf{A}x - x|_1$ para cada uno de los grafos sobre cada valor de $p$ en en el intervalo $(0, 1)$ de a saltos de $0.01$.
\vspace{1em}

\noindent Cada caso, representable por una matriz de conectividad W $\in \{0,\ 1\}^{100 \times 100}$, se generó a través del siguiente procedimiento\footnote{Se utilizó un valor semilla para facilitar la reproducibilidad.}:
\vspace{1em}

\begin{enumerate}
    \item Se eligió  la cantidad de ejes (e) del sistema de manera uniforme sobre el intervalo $[0,\ R \cdot T)$, donde $T = 100^2 - 100$ representa el máximo de ejes posibles en un grafo de cien nodos sin auto-direccionamiento y $R = 1/4$ es un valor arbitrario definido para imitar las carácteristicas de ralidad esperables en un conjunto de páginas web.
    \item Se pobló una matriz $W_0 \in \{0,\ 1\}^{99 \times 100} = 0$ con unos en las primeras `e' posiciones y se utilizó el algoritmo de shuffle de numpy, sobre el rng \textsc{PCG64}, para generar una permutación aleatoria.
    \item Se expandió la misma con ceros en la diagonal para lograr la matríz $W \in \{0, 1\}^{100 \times 100}$. 
\end{enumerate}
\vspace{1em}


\noindent \textsc{Observaciones}. El experimento tiene como limitaciones principales el tamaño de la muestra (cien grafos distintos) y el método de generación de casos ---los mismos no provienen de muestras reales---, que incluye la elección arbitraria del valor $R$. Sin embargo, contempla con cierta granularidad todo el expectro de valores posibles para $p$, tal que permite conocer el error relativo de los resultados en función de su parámetro `libre'.
\vspace{1em}


\noindent \textsc{Resultados}. El cuadro 1. resume los resultados obtenidos. 
\vspace{1.5em}

\begin{center}
    \begin{tabular}{ |l|c| } 
     \hline
    mediciones              & 10000         \\
    error relativo promedio & 0.00084646    \\
    desviación estándar     & 0.00063764    \\
    mínimo                  & 1.7347e-16    \\
    25\%                    & 0.00034960    \\
    50\%                    & 0.00075431    \\
    75\%                    & 0.00121044    \\
    máximo                  & 0.00376671    \\
    \hline
    \end{tabular} \\
    \bigskip
    Cuadro 1. datos de resumen del experimento. 
\end{center}
\vspace{1em}

Podemos observar que el error relativo fue en promedio menor a $1e-3$. De distribuirse uniformemente, esto nos permite suponer que el error relativo de cada puntaje debe estar en el orden de $1e-5$.
\vspace{1em}

La Figura \ref{img_error_relativo}. muestra la distribución de los resultados en función del parámetro $p$.

\fig[]{files/src/media/error_relativo.png}{Error relativo promedio -en base a una norma L1- e intervalo de confianza del 95\% para una muestra aleatoria de cien grafos en función de $p$. }{img_error_relativo}{}
\vspace{1em}

Notar la progresiva disminución del error a medida que $p$ se aproxima a 1. Los máximos locales se encuentran en $p = 0.16$ (0.00101828)  y $p = 0.55$ (0.00117609). El mínimo local en $p = 0.26$ (0.00056232)\footnote{El cálculo de los picos locales se puede encontrar en $./experimentos/resultados/error\_relativo$}. 

\vspace{1em}
Si bien el tamaño de la muestra y su método de generación no permiten sacar conclusiones fuertes al respecto de los resultados, sí podemos notar que el valor de $p$ influencia el error relativo. Desde un punto de vista numérico esto tiene sentido, dado que $p$ reduce la  magnitud de los valores sobre los que trabaja el algoritmo. Sin embargo, la forma particular de la distribución del error en función de $p$ resultó sorpresiva. 

\vspace{1em}
A modo de consideración para futuros análisis, podemos mencionar que nuestro método de `corte' ---cualquier valor menor a $1e-4$ se anula--- también debe influir en el error de los resultados. 


\newpage
\subsubsection{Error absoluto} Medimos el error $|x - \hat{x}|_1$ para los casos de test provistos por la cátedra\footnote{El script asociado se puede encontrar en $./experimentos/error\_tests.py$. Los resultados coordenada a coordenada se pueden observar en $./experimentos/resultados/error\_tests$.}. 
\vspace{1em}

\noindent El Cuadro 2. resume los resultados.
\vspace{1.5em}

\begin{center}
    \begin{tabular}{ |l|c|c| } 
    \hline
    test                         & $L_1$        & $L_\infty$ \\
    \hline
    test\_15\_segundos           & 0.0291137    & 5.015e-05\\
    test\_30\_segundos           & 0.0229572    & 4.841e-05\\
    test\_aleatorio              & 6.176e-07    & 2.518e-07\\
    test\_aleatorio\_desordenado & 6.176e-07    & 2.518e-07\\
    test\_completo               & 0.0          & 0.0\\
    test\_sin\_links             & 0.0          & 0.0\\
    test\_trivial                & 0.0          & 0.0\\
    \hline
    \end{tabular} \\
    \bigskip
    Cuadro 2. Error absoluto en base a la norma $L_1$ y $L_\infty$ de los tests.
\end{center}
\vspace{1em}

Podemos observar que hay cierta correlación entre la cantidad de páginas y el error. Esto tiene sentido dado que la norma L1 es la suma del valor absoluto de las coordenadas. Mientras mayor sea la dimensión, mayor será la cantidad de errores a sumar.




\newpage
\subsection{Análisis cualitativo} Para entender en profundidad las propiedades de $PageRank$, decidimos evaluar distintos escenarios `simples' de grafos. Comenzamos por los casos triviales.




\vspace{2em}
\subsubsection{Escenario 1: Sin Links} Ninguna página tiene un vínculo a ninguna otra. Este es el caso base que el \textit{modelo del navegante aleatorio} busca resolver respecto a modelos anteriores. 

\vspace{1em}
\fig[]{files/src/media/conceptuales/sin_links.png}{Una web con seis páginas desvinculadas.}{sin_links}{0.20}

\vspace{1em}
\noindent \textsc{Metodología}. Se generaron cien conjuntos de páginas web sin links donde se varió la cantidad total de sitios (de uno a cien). Se midió el puntaje de cada uno en los rankings obtenidos. 

\vspace{1em}
\noindent \textsc{Resultados}. Observamos que el puntaje de todas las páginas fue igual en cada caso, pero fue decrementando en función de la cantidad total de nodos. 

\vspace{1em}
\fig[]{files/src/media/sin_links.png}{El puntaje de una página testigo en función de la cantidad de páginas para conjuntos sin links.}{resultado_sin_links}{}

\vspace{1em}
El puntaje obtenido para cada página fue de $\frac{1}{n}$. La figura \ref{resultado_sin_links}. grafica los resultados para un caso testigo. Esto se explica teóricamente: si no hay links en la web, entonces la matriz de conectividad $\mathbf{W}$ es nula y $\mathbf{I} - p\mathbf{W}\mathbf{D} = \mathbf{I}$. Como $\mathbf{I}x = 1 \implies x_i = 1$. Al normalizar nos queda que $x_i = \frac{1}{n}\ \forall i: 1\ ...\ n$. 

\vspace{1em}
El resultado es coherente con la interpretación intuitiva del ranking: si ninguna página redirige a ninguna otra, entonces un navegante sólo podrá acceder a los sitios del conjunto de manera aleatoria. El modelo define esta probabilidad de manera uniforme sobre el total de las páginas. A mayor cantidad de páginas, menor la probabilidad que el navegante aleatorio termine en un sitio en particular.









\vspace{2em}
\subsubsection{Escenario 2: Simetría} Otro escenario trivial es en el que todas las paginas tienen links a todas las demás. 

\vspace{1em}
\fig[]{files/src/media/conceptuales/todos_con_todos.png}{Una web simétrica dónde cada nodo apunta al resto.}{todos_con_todos}{0.26}

\vspace{1em}
\noindent \textsc{Metodología}. Para analizar esta situación, se generaron cien casos distintos donde se varió la cantidad de nodos (entre uno y cien) y se mantuvo la simetría de las relaciones. En el primero caso, se compuso una web con un único nodo. Luego, con dos nodos apuntándose mutuamente y así progresivamente hasta llegar a cien nodos donde cada uno apuntó al resto. Se utilizó un valor de $p$ de 0.85\footnote{Según \cite{Langville04} éste es el valor que utiliza Google.}.

\vspace{1em}
\noindent \textsc{Resultados}. Observamos que el puntaje de cada una de las páginas fue igual para cada caso particular\footnote{Se puede ver el resultado de los experimentos en $./experimentos/resultados/todos\_con\_todos$.} pero que este valor fue decrementando a medida que aumentó la cantidad de sitios. Luego de observar el resultado notamos que para cada $n$ el valor resultante fue exactamente el mismo que en caso "sin links".

Esto tiene sentido ya que si todos los nodos se apuntan entre sí, el sistema es totalmente simétrico, y sería ilógico que alguno tenga más importancia que los demás. 

\vspace{1em}
\noindent Pudimos notar la misma clase de relación, y los exactos mismos resultados, con otras estructuras simétricas. Por ejemplo, en estructuras de referencias circulares\footnote{Ver $./experimentos/resultados/circular$}.


\newpage
\subsubsection{Escenario 3: Todos con uno} Para esta estructura, tal y como su nombre indica, se enlazó a todas las páginas en el conjunto con una única otra. Se referirá a ésta como la página \textit{'uno'}.

\vspace{1em}
\fig[]{files/src/media/conceptuales/todos_a_uno_3.png}{Una web centralizada, donde todos los nodos señalan a la misma página.}{todos_con_uno}{0.26}

\vspace{1em}
\noindent \textsc{Metodología}. Se generaron cien casos distintos donde se varió la cantidad de enlaces que conectan con el \textit{'uno'} y se midió su puntaje y el de un segundo nodo testigo. Se controló la cantidad de nodos (cien) y se evitaron otros vínculos. Se usó un valor de $p$ de 0.85.

\vspace{1em}
\noindent \textsc{Hipótesis}. Consideramos que el puntaje se comportará de la siguiente forma:
\begin{itemize}
\item El \textit{'uno'} tendrá un ranking muy cercano a 1, que aumentará de tamaño de manera proporcional a la cantidad de páginas existentes. Estimamos que el crecimiento adoptará una curva similar a una logarítmica.
\item Las demás páginas tendrán un ranking muy cercano a 0, que disminuirá en relación al numero de enlaces existentes. %Estimamos que el decrecimiento adoptará una curva similar a $\frac{1}{n}$.
\end{itemize}

\vspace{1em}
\fig[]{files/src/media/todos_a_uno.png}{El puntaje del nodo `uno' y de otro sitio testigo para el conjunto, en función de la cantidad de referencias que recibe `uno'.}{todos_a_uno}{}

\vspace{1em}
\noindent \textsc{Resultados}. Observamos que los resultados obtenidos distan un poco de los especulados\footnote{Se puede ver el resultado de los experimentos en $./experimentos/resultados/todos\_a\_uno$.}, ya que, si bien el decrecimiento del resto de las páginas fue de la forma descrita (es difícil observar en el gráfico), el comportamiento del \textit{`uno'} no se comportó de manera logarítmica. Consideramos que es más parecida a alguna variante de $\sqrt{n}$. Esto nos llamó la atención.











\vspace{2em}
\subsubsection{Escenario 4: Uno con todos} Para esta estructura, se propuso el caso inverso al anterior: una única página, que será referenciada nuevamente como \textit{'uno'}, tendráenlaces salientes hacia todas las demás. Además, cualquier otro tipo de relación será nula. Los enlaces de \textit{'uno'} serán únicamente salientes.

\vspace{1em}
\fig[]{files/src/media/conceptuales/uno_a_todos_3.png}{Una web muy simétrica, donde todos, excepto \textit{'uno'} poseen el ranking máximo.}{uno_con_todos}{0.26}

\vspace{1em}
\noindent \textsc{Metodología}. Para analizar esta situación, se generaron cien casos distintos donde se varió la cantidad de enlaces salientes de \textit{'uno'}. De esta forma, veremos la evolución de una web a medida que el \textit{'uno'} va conectando con las demás páginas. Para evaluar los resultados de la evolución, graficamos las variaciones de tres páginas. La \textit{'uno'}, la primer página con la que conecta y la última con la que lo hace. Así, podremos ver la evolución desde varias perspectivas.

\vspace{1em}
\noindent \textsc{Hipótesis}. Suponemos que se comportará de la siguiente forma:
\begin{itemize}
\item La \textit{'uno'} tendrá un ranking significativamente menor que todas las demás páginas, disminuyendo a medida que aumenta el número de páginas. Estimamos que el decrecimiento adopte una curva similar al $\frac{1}{n}$.
\item Las demás páginas irán disminuyendo su ranking hasta cierto punto, donde alcanzarán un máximo, cuando \textit{'uno'} las conecte. Luego, seguirán disminuyendo su ranking hasta que todas posean el mismo, excepto \textit{'uno'}.
\end{itemize}

\vspace{1em}
\fig[]{files/src/media/uno_a_todos.png}{El puntaje del nodo `uno' y de dos otros sitios testigos para el conjunto, en función de la cantidad de referencias saliente de `uno'.}{uno_a_todos}{}

\vspace{1em}
\noindent \textsc{Resultados}. Observamos que los resultados difieren en gran forma de lo especulado para lo que es el \textit{'uno'}\footnote{Se puede ver el resultado de los experimentos en $./experimentos/resultados/uno\_a\_todos$.}, la disminución de su ranking fue abrupta ni bien se conectó el primer enlace. Por otro lado, las suposiciones hechas para los demás son tal cual se esperaban. Fue sorpresiva la cantidad de puntaje que recibió el primer testigo al ser el único link referenciado.




\newpage
$\\$
\subsubsection{Escenario 5: Transitividad en Cadena} Cómo se transfiere el puntaje de un sitio a otro a medida que se extiende la cadena de páginas intermedias entre ambos.

\vspace{1em}
\fig[]{files/src/media/conceptuales/transitividad_en_cadena_3.png}{Una cadena de nodos.}{cadena}{0.4}

\vspace{1em}
Consideramos de interés ver cómo se ve afectado el puntaje de una página al ser apuntada por otra `importante', y cómo varía a medida que empiezan a haber más intermediarios en la relación. 

\vspace{1em}
\noindent \textsc{Metodología}. Se generaron cien instancias de test con 200 nodos cada una ---para controlar las variaciones del puntaje relacionadas a la inserción de nuevos nodos\footnote{Como se comprobó sucede en el Escenario 1.}--- y se estableció que las páginas indexadas en el rango (100, 200] apunten a la página $1$ y que el resto esté desvinculada. Se utilizó un parametro $p$ de 0.85.

\vspace{1em}
\fig[]{files/src/media/cadena.png}{El puntaje del último eslabón en una cadena de vínculos, en función de su largo. Salvo la primer página, el resto de las páginas en la cadena no tiene otras relaciones.}{resultado_cadena}{}

\vspace{1em}
Se procedió a generar una cadena cuyo tamaño se incrementó en uno en cada instancia sucesiva del test: en la primera, el nodo `1' apuntó al nodo `2'; en la segunda, se extendió la cadena y el nodo `2' apuntó al nodo `3'; se procedió de igual forma hasta llegar al último nodo desvinculado. En cada instancia, se midió el puntaje del último nodo en la cadena.

\vspace{1em}
\noindent \textsc{Resultados}. Como se ve en la Figura \ref{resultado_cadena}. el puntaje del último eslabón se hace más chico a medida que la cadena se hace más larga. De igual manera, el resultado se acerca progresivamente a $1/l$, donde $l$ es el largo de la cadena. 

\vspace{1em}
Podemos ver que la influencia de la página `importante' es fuerte al principio, pero se diluye rápidamente en la cadena. Consideramos que el comportamiento es deseable: Si una página importante apunta a otra, entonces es esperable que aumente la probabilidad que un navegante decida acceder a éste segundo sitio. Pero si agregamos un tercer eslabón en la cadena, entonces la probabilidad que se acceda a ésta tercera página debería ser la intersección de las probabilidades anteriores, y en consecuencia más improbable. 





\vspace{2em}
\subsubsection{Escenario 6: Referencias Valiosas} Cómo varía el peso de una página a medida que cambia la importancia de las páginas que la referencian y el valor de $p$.

\vspace{1em}
\noindent \textsc{Metodología}. Se generaron cien instancias de 102 nodos cada una y se simuló la siguiente situación: inicialmente, el nodo `1' apunta al nodo `2' y todos los demás nodos están aislados. En cada iteración del test se selecciona uno de los nodos aislados, se lo vincula al nodo `1' (en esta misma dirección) y se mide el puntaje de `1' y de `2'. Se repitió el experimento para cuatro valores de $p$ (0.25, 0.50, 0.75 y 0.9999).

\vspace{1em}
\fig[]{files/src/media/conceptuales/transitividad_en_paralelo_3.png}{La influencia de un nodo sobre otro, a medida que cambia su importancia.}{paralelo}{0.18}

\vspace{1em}
\noindent \textsc{Hipótesis} Inicialmente el peso del nodo `2' va a ser mayor que el del nodo `1'. Nos basamos en la observación que si el nodo `1' no tiene páginas que la referencien y el `2' tiene una referencia, entonces va a ser mayor. A medida que más nodos apunten a `1', el peso de éste va a superar al del nodo `2'\footnote{Dados los resultados que obtuvimos en el escenario `Transitividad en Cadena'.}. De todas maneras, intuímos que la intensidad de está relación dependerá del valor de $p$.

\vspace{1em}
\noindent \textsc{Resultados}. Se cumplieron nuestras espectativas. Como se puede apreciar en las Figuras \ref{res_paralelo_1}. a  \ref{res_paralelo_4}. el valor de $p$ influenció cómo se transfirió el puntaje de un nodo al otro. 

Concluímos que es esencial tener referencias valiosas para aumentar el puntaje en el ranking. Sin embargo, la relación de transferencia no es lineal y depende de $p$. Notamos que las páginas con mayor puntaje son también más influyentes y pueden generar cambios radicales en las valoraciones del algoritmo.

\vspace{1em}
\fig[]{files/src/media/referencias_valiosas_p0.25.png}{El puntaje de los nodos `1' y `2', donde el primer sitio vincula al segundo, en función del puntaje del primero, para $p = 0.25$.}{res_paralelo_1}{}

\vspace{1em}
\fig[]{files/src/media/referencias_valiosas_p0.5.png}{El puntaje de los nodos `1' y `2', donde el primer sitio vincula al segundo, en función del puntaje del primero, para $p = 0.5$.}{res_paralelo_2}{}

\vspace{1em}
\fig[]{files/src/media/referencias_valiosas_p0.75.png}{El puntaje de los nodos `1' y `2', donde el primer sitio vincula al segundo, en función del puntaje del primero, para $p = 0.75$.}{res_paralelo_3}{}

\vspace{1em}
\fig[]{files/src/media/referencias_valiosas_p0.9999.png}{El puntaje de los nodos `1' y `2', donde el primer sitio vincula al segundo, en función del puntaje del primero, para $p = 0.9999$.}{res_paralelo_4}{}

\vspace*{10cm}
% \newpage
% $\\$
% \subsubsection{Escenario 7: Referenciador Importante} Cómo varía el peso de un nodo importante a medida que referencia a más nodos. 

% \vspace{1em}
% \noindent \textsc{Metodología}. Se generaron 25 casos aleatorios de cien nodos cada uno. Simulación: Inicialmente todos los nodos apuntan a `1' y en cada iteración se agrega un vínculo de `1' a alguno de los otros nodos. En cada una de estas iteraciones almacenamos el puntaje de `1'.

% \vspace{1em}
% \noindent \textsc{Resultados}. El resultado no fue el esperado, ya que como se ve en el gráfico la curva de color xxx, el peso de `0' de no referenciar a nadie a referenciar a un nodo baja pero luego se mantiene constante, como si el impacto que genera en el peso de un nodo particular referenciar a una cantidad arbitraria de nodos fuera comparable con referenciar a un único nodo. 

% Pensamos que quizás en un ambiente más `real' con relaciones aleatorias entre los demás nodos del sistema podría alterar nuestros resultados, por ello reestructuramos el test para que además de las relaciones previamente mencionadas los nodos [1 a 20] estén conectados entre sí de manera aleatoria. Para obtener resultados significativos que no esten `contaminados' por la aleatoreidad corrimos el test 10 veces, sumamos el resultado de cada una de las iteraciones y calculamos su promedio.
% El resultado obtenido fue similar al resultado previo por ende deducimos que no referenciar a ningún nodo es la estrategia óptima y que a partir de referenciar a uno o más nodos en un ambiente aleatorio genera, en promedio, el mismo impacto en el peso de la la página que referencia.
