El Ranking de Page, \textit{PageRank \cite{Brin98}}, es un método propuesto por Sergey Brin y Larry Page ---co-fundadores de Google---, para jerarquizar las páginas web del internet, o un subconjunto de las páginas que lo componen. Holísticamente, se puede interpretar como una medición, al largo plazo, del porcentaje de tiempo que un navegante permancerá en cada uno de los sitios \cite{Bryan06}. 

Desde una perspectiva algorítmica, \textit{PageRank} busca resolver un sistema lineal $\textbf{A}x = x$, donde \textbf{A} es una matriz estocástica en columnas \cite{Bryan06} y cada una de sus posiciones $a_{ij}$ representa la probabilidad que un usuario situado en la página $j$ decida navegar a la página $i$. 

Este trabajo propone una implementación eficiente del ranking a través del uso de una estructura de matríz acorde al problema, y el empleo de iteradores específicos, para reducir el costo espacial y temporal de la eliminación gaussiana, método utilizado para su resolución. 

Se buscará dar una presentación teórica y una evaluación cuantitativa y cualitativa de los resultados de tanto el método propuesto, como de \textit{PageRank} en si.  

$\\$
\noindent Palabras clave: \textit{Ranking de Page, Eliminación Gaussiana, Representación de Matrices}
