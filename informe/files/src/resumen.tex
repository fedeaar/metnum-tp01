El Ranking de Page, \textit{PageRank \cite{Brin98}}, es un método propuesto por Sergey Brin y Larry Page ---co-fundadores de Google---, para establecer la importancia de una página web dentro del internet, o dentro de un subconjunto de las páginas que lo componen. Holísticamente, la relevancia de un sitio se interpretará como la fracción del tiempo, al largo plazo, que un navegante permanecerá allí \cite{Bryan06}. 

Desde una perspectiva algorítmica, PageRank busca resolver un sistema lineal $\textbf{A}x = x$, donde \textbf{A} es una matriz estocástica en columnas \cite{Bryan06} y cada una de sus posiciones $a_{ij}$ representa la probabilidad que un usuario situado en la página $j$ decida navegar a la página $i$. 

Este trabajo propone una implementación eficiente del ranking a través del uso de una estructura de matríz acorde al problema, y el empleo de iteradores específicos, para reducir el costo espacial y temporal de la eliminación gaussiana, método utilizado para la resolución del sistema. 

Se buscará dar una presentación teórica y una evaluación cuantitativa y cualitativa de los resultados de tanto el método propuesto, como de PageRank en si.  

$\\$
\noindent Palabras clave: \textit{Ranking de Page, Eliminación Gaussiana, Matrices ralas}
