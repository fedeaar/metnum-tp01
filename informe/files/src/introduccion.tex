\vspace{1em}

% === aridad === %
\subsection{Aridad} Consideremos primero el dominio y la imágen de PageRank. 

\vspace{1em}
\noindent \textsc{Dominio}: \textsc{1.} un conjunto de páginas web interconectadas a través de hipervínculos. Podemos considerar este conjunto como un grafo direccionado, donde los nodos son los sitios y los ejes, los links. \textsc{2.} un parámetro de entrada $p \in (0,\ 1)$, que representa la probabilidad que un usuario decida navegar aleatoriamente a otra página en el grafo. Se puede interpretar como el parámetro de un variable aleatoria de Bernoulli. 

\vspace{1em}
\noindent \textsc{Imágen}: un vector $x \in [0,\ 1]^{n}$, donde $x_i$ representa el Ranking de Page para la i-ésima página del conjunto de entrada, donde $x$ satisface que $\ x_i \geq 0\ \forall i:0\ ...\ n\ $ y $\ \sum_{i=1}^{n}{x_i} = 1$.

\vspace{1em}
\noindent Tenemos entonces:

\begin{equation}
    PageRank:\ G_n \times (0,\ 1) \longrightarrow [0,\ 1]^{n}\qquad \forall n \in \mathbb{N}
\end{equation}

\vspace{1em}
\noindent donde $G_n$ refiere al conjunto de conjuntos de páginas web, interconectadas a través de hipervínculos, con cardinalidad $n$.




% === el sistema === %
\vspace{2em}
\subsection{El sistema} PageRank propone resolver un sistema de ecuaciones para encontrar la relevancia de cada página $i$ $(i: 1\ ...\ n)$ en $g \in G_n$:
\vspace{1em}
\begin{equation}
    x_i := \sum_{j=1}^{n} x_j \cdot Pr(j \longrightarrow i) 
\end{equation}

\vspace{1em}
\noindent donde $Pr(j \longrightarrow i)$ es la probabilidad que un usuario situado en la página $j$ decida ir a la página $i$. Se define de la siguente manera:

\vspace{1em}
\begin{equation}
    Pr(j \longrightarrow i) := 
        \left\{ 
            \begin{array}{lcc}
            (1 - p)\cdot \frac{1}{n} + p\cdot \frac{I_{ij}}{c_j}    &  \text{si}    & c_j \neq 0\\
            \frac{1}{n}                                             &  \text{si no} &
            \end{array}
        \right.
\end{equation}

\vspace{1em}
\noindent con $\ I_{ij} = 1\ $ si y sólo si existe un hipervínculo de $j$ a $i$, con $j \neq i$ ---y nulo en caso contrario---, y $\ c_j = \sum_{i=1}^{n}{I_{ij}}\ $, la cantidad de links salientes de $j$. La restricción $j \neq i$ será para evitar que se consideren auto-referencias en el ranking.    

\vspace{1em}
Notemos que $Pr(j \longrightarrow i)$ se puede interpretar de la siguente manera: un navegante situado en la página $j$ decidirá con probabilidad $p$ acceder a uno de los links del sitio y con probabilidad $1 - p$ saltar a otra página del conjunto. En ambos casos, deberá luego decidir uniformemente sobre el total disponible, y terminará eligiendo a $i$ con una probabilidad de $\frac{I_{ij}}{c_j}$ ó $\frac{1}{n}$, respectivamente, acorde a la primer decisión. Si no hay links en la página, siempre saltará de manera uniforme a otra página del conjunto, y eligirá a $i$ con probabilidad $\frac{1}{n}$. 

\vspace{1em}
$x_i$, por su parte, también recibe una interpretación particular: es la probabilidad que para algún momento $k > K$, el navegante se encuentre situado en la página $i$. Para un $K$ lo suficientemente grande, esta probabilidad es única \cite{Kamvar03}. 

\vspace{1em}
A este modelo se lo conoce como el \textit{modelo del navegante aleatorio}. El mismo asume lo siguiente: un link de la página $j$ a la página $i$ es evidencia de la importancia de la página $i$. Específicamente: la cantidad de relevancia que le confiere la página $j$ a la página $i$ es proporcional a la relevancia de $j$ e inversamente proporcional al número de páginas a las que apunta $j$ \cite{Kamvar03}. Esto se puede ver en que $x_i$ es la suma de toda otra importancia $x_j$ ponderada por $Pr(j \longrightarrow i)$, que incluye en su definición una división sobre la cantidad de links salientes $c_j$. 



% === representación matricial === %
\vspace{2em}
\subsection{Representación matricial} Dado que estamos trabajando con un sistema lineal, será de utilidad considerar la matriz asociada \textbf{\textit{A}} y resolver, equivalentemente, $\textbf{A}x = x$. Definimos entonces:

\begin{equation}
    a_{ij} := Pr(j \longrightarrow i)
\end{equation}

\vspace{1em}
\noindent y proponemos que\footnote{Una demostración de esta equivalencia se encuentra en \ref{A.1}.}:

\begin{equation}
    \textbf{A} = p\textbf{W}\textbf{D} + ez^t
\end{equation}

\vspace{1em}
\noindent donde, $\forall i,\ j:\ 1\ ...\ n$, se satisface que:

\begin{align*}
    e_i     &=  1
    \\
    \\
    z_{j}   &=  \left\{ 
                    \begin{array}{lcc}
                    (1 - p) / n     &  \: \text{si}    &  c_j \neq 0 \\
                    1 / n           &  \: \text{si no} &
                    \end{array}
                \right.\
    \\
    \\
    w_{ij}  &=  \left\{ 
                    \begin{array}{lcc}
                    1               &  \qquad \qquad \text{si}    & i \neq j\  \wedge\ j \stackrel{l}{\longrightarrow} i \\
                    0               &  \qquad \qquad \text{si no} &
                    \end{array}
                \right.\
    \\
    \\
    d_{ij}  &=  \left\{ 
                    \begin{array}{lcc}
                    1 / c_j         &  \qquad \: \: \text{si}    & i = j\  \wedge\ c_j \neq 0 \\
                    0               &  \qquad \ \  \text{si no} &
                    \end{array}
                \right.\
\end{align*}
\vspace{1em}

\noindent La notación $j \stackrel{l}{\longrightarrow} i$ representa que existe un link de la página $j$ a la página $i$, y las filas y columnas de $\textbf{W}$, denominada \textit{matriz de conectividad}, representan ---indexadas por posición--- las páginas de $g$.
\vspace{1em}

\noindent A partir de esta última equivalencia podemos ver que:

\begin{align*}
    \textbf{A}x &= x \\
    (p\textbf{W}\textbf{D} + ez^t)x &= x \\
    p\textbf{W}\textbf{D}x + ez^tx &= x \\
    x - p\textbf{W}\textbf{D}x &= ez^tx \\
    (\textbf{I} - p\textbf{W}\textbf{D})x &= \gamma e \\
\end{align*}

\noindent donde $\gamma = z^tx$.
\vspace{1em}

\noindent Si asumimos $\gamma = 1$, entonces el sistema a resolver será:
\vspace{1em}

\begin{equation}\label{sistema}
    (\textbf{I} - p\textbf{W}\textbf{D})x = e
\end{equation}
\vspace{1em}

El sistema definido en (\ref{sistema}) permite la aplicación de la eliminación gaussiana sin permutación.\footnote{Una demostración de este enunciado se encuentra en \ref{A.2}.} Por lo que utilizaremos éste método para su resolución. Como no necesariamente $\ \sum_{i=1}^{n}{x_i} = 1$, normalizaremos el resultado alcanzado para satisfacer los requerimientos de la imágen.
