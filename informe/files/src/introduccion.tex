\vspace{1em}

\subsection{Aridad}
$\\\\$
Consideremos primero el dominio y la imágen del problema. Tenemos un conjunto de páginas web conectadas unas a otras por hipervínculos. Como única condición vamos a ignorar las auto-referencias. Si nos abstraemos, podemos considerar que nuestro dominio es el conjunto de todo grafo direccionado sin auto-direccionamiento (los nodos son los sitios y los ejes, los links). En consecuencia, podemos proponer la siguiente representación:

\vspace{0.25em}
\begin{equation}
W_n = \{w_n\ |\ w_n \in \{0,\ 1\}^{n \times n},\ (w_n)_{ij} =    
    \left\{ 
        \begin{array}{lcc}
        1           &  \text{si}    & i \neq j\  \wedge\ j \stackrel{l}{\longrightarrow} i \\
        0           &  \text{sino}  &
        \end{array}
    \right.\ \forall i,\ j:\ 1\ ...\ n\ \}
\end{equation}
\vspace{0.5em}

\noindent donde la notación $j \stackrel{l}{\longrightarrow} i$ representa un link de la página $j$ a la página $i$, y las filas y columnas de $w_n$, denominada \textit{matríz de conectividad}, representan, indexadas por posición, las páginas del grafo.  

\vspace{1em}
$PageRank$ define, además, un parámetro de entrada $p \in (0,\ 1)$, que representa la probabilidad de que un usuario decida navegar aleatoriamente a otra página en el grafo. Notar que $p$ se puede interpretar como el parámetro de un variable aleatoria bernoulli, y que no está definida para sus extremos.

\vspace{1em}
La imágen, por su parte, es un vector $\vec{s} \in [0,\ 1]^{n}$, donde $\vec{s}_i$ representa el Ranking de Page para la i-ésima página del grafo de entrada. Los valores están normalizados, tal que:

\begin{equation*}
\sum_{i=0}^{n}{\vec{s}_i} = 1
\end{equation*}

\vspace{1em}
\noindent Tenemos entonces:

\begin{equation}
    PageRank:\ W_n \times (0,\ 1) \longrightarrow [0,\ 1]^{n}\qquad \forall n \in \mathbb{N}
\end{equation}


\vspace{1em}

\subsection{Matriz de probabilidad}
$\\\\$
La resolución de PageRank requiere resolver un sistema de ecuaciones lineales sobre una matriz de probabilidades. Esta se define de la siguiente manera: